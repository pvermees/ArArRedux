\documentclass{article}
\usepackage{natbib,graphicx,fullpage}
\begin{document}

\setcounter{section}{4}
\setcounter{subsection}{0}
\subsection{Reporting \textsuperscript{40}Ar/\textsuperscript{39}Ar uncertainties}
\label{sec:reportinguncertainties}

\subsubsection{Assessing goodness of fit: pitfalls and opportunities}
\label{sec:MSWD}

The random scatter of the data around an isochron or weighted mean fit
can be assessed using the Mean Square of the Weighted Deviates (MSWD,
McIntyre et al., 1960). This statistic is more generally known as the
`reduced Chi-square statistic' outside geology. The MSWD is defined as
the sum of the squared differences between the observed and the
expected values, normalised by the analytical uncertainties and
divided by the degrees of freedom of the fit.  In the context of the
weighted mean age, the MSWD of $n$ values is given by:

\begin{equation}
  \mathrm{MSWD} = \frac{1}{df} \sum\limits_{i=1}^{n} \frac{\left(x_i
    - \bar{x}\right)^2}{\sigma_i^2}
  \label{eq:MSWD}
\end{equation}

\noindent where $x_i$ is the $i$\textsuperscript{th} (out of $n$)
dates, $\sigma_i$ is the corresponding analytical uncertainty, $df = n
- 1$ is the number of degrees of freedom, and $\bar{x}$ is the
weighted mean of all $n$ dates. The definition for the MSWD of an
isochron is similar but has one fewer degree of freedom ($df = n - 2$)
and involves a few more terms to account for correlated uncertainties
between the x- and y-variable. The following are general MSWD
considerations:

\begin{enumerate}
\item If the analytical uncertainties $\sigma_i$ are the only source
  of scatter between the $n$ aliquots, and $df$ is reasonably large
  (for $n > 20$, say) then MSWD$\approx$1 (Fig.~5A,B).  For smaller
  sample sizes, the MSWD has a much wider distribution with an
  expected value of less than one (Wendt and Carl, 1991; Mahon
  1996). Figure~1 of the Supplementary Information shows the
  probability distribution of the MSWD for different sample sizes.
  The remainder of this section will assume that $n > 20$.

\item MSWD values that approach zero indicate that analytical
  uncertainties have been overestimated or have not been propagated
  correctly (Fig.~5B,C). Assigning ages to samples based on such
  \emph{underdispersed} data should be done with caution.
  
\item MSWD-values considerably greater than one indicate that there is
  some excess scatter in the data, which cannot be explained by the
  analytical uncertainties alone. This may reflect underestimation of
  analytical uncertainties, but usually reflects the presence of some
  geological \emph{(over)dispersion} affecting the dataset and/or
  neutron fluence gradients. Possible causes of such dispersion may
  include the protracted crystallization history of a sample, variable
  degrees of inheritance, or partial loss of radiogenic
  \textsuperscript{40}Ar by retrograde reactions, thermally activated
  volume diffusion, deformation, or chemical alteration (Fig.~D,E).
\end{enumerate}

It would be wrong to believe that only datasets with MSWD$\approx$1
are suitable for publication. Trimming an overdispersed dataset by
selectively rejecting outliers until achieving an MSWD$\approx$1 is
likely ill-advised as this risks the loss of geologically valuable
information and biasing the results. Outlier identification and
rejection must always be accompanied by full disclosure of the
specific criteria used for such evaluation, and not simply to improve
the statistics of a dataset. High MSWD values do not necessarily
indicate poor data. In fact, with modern-day high-precision mass
spectrometery, a higher MSWD may simply reflect the achieved higher
analytical precision of the data. Increasingly dispersed datasets are
likely to become even more prevalent in the future, as a result of the
ever-increasing improvements of mass spectrometers. In this case, the
excess dispersion can be formally assessed with a Chi-square test for
homogeneity, and its associated p-value, in the case the experiments
were carried out with large number of heating steps or total fusions
of single crystals. However, unpowered (sensu Cohen, 1992) statistical
hypothesis tests have come under criticism in recent years, and
scientists are increasingly advised not use them (Wasserstein and
Lazar, 2016; Amrhein et al., 2019).\\

Dispersed datasets should always be evaluated carefully on a case by
case basis, and any conclusions based on dispersed data should be done
with caution. It is important to attempt to consider the potential
causes (geologic, analytical, mineralogic, etc.) of the data
dispersion (e.g., Verati and Jourdan, 2014; Phillips et al., 2017). In
some cases, a subset of a dispersed dataset can be used to assign an
age for a sample given sufficient geologic context. For example,
single crystal fusion dates from a volcaniclastic layer intercalated
within a fluvio-lacustrine succession along the Tiber River, Italy
show significant dispersion (MSWD = 603; Marra et al., 2019). The
volcaniclastic layer has lithologic and mineralogic characteristics
that are nearly identical to another volcaniclastic layer located ~ 6
km to the northwest that was dated at 327.5 $\pm$ 3.5 ka. The youngest
six \textsuperscript{40}Ar/\textsuperscript{39}Ar dates of the
dispersed dataset give a weighted mean age of 328.7 $\pm$ 1.6 ka,
which led Marra et al. 2019 to conclude that the two dated
volcaniclastic layers are indeed identical and have been tectonically
displaced by 50 meters.\\

When no potential sources of data dispersion can be confidently
identified, it can be assumed that the excess dispersion is
multiplicative and scales in proportion to the analytical
uncertainty. In this case, the standard error of the weighted mean or
isochron intercept may be augmented by multiplying it with the square
root of the MSWD (Ludwig, 2003). A second option is to parameterize
the overdispersion as an additive term and estimate it as a separate
parameter (Galbraith and Laslett, 1993; Vermeesch, 2018).\\

\textbf{Additional references:}\\

Cohen, J., 1992. A power primer. Psychological bulletin, 112(1),
p.155.\\

Galbraith, R.F. and Laslett, G.M., 1993. Statistical models for mixed
fission track ages. Nuclear tracks and radiation measurements, 21(4),
pp.459-470.\\

Ludwig, K.R., 2003. Mathematical–statistical treatment of data and
errors for \textsuperscript{230}Th/U geochronology. Reviews in
Mineralogy and Geochemistry, 52(1), pp.631-656.\\

\textbf{Supplementary Figure:}\\

\begin{figure}
  \centering
  \includegraphics[width=0.7\textwidth]{MSWDdistribution.pdf}
  \caption{Probability distribution for the Mean Square of the
    Weighted Deviates (MSWD) for different degrees of freedom. Small
    samples ($df<20$) are characterised by skewed and broadly
    distributed MSWD distributions whose mean, median and mode are all
    less than 1. With increasing sample size ($df>20$), the
    probability distribution of the MSWD becomes more symmetric
    converges to an expected value of MSWD$\approx$1.  }
\end{figure}


%\bibliographystyle{/home/pvermees/Dropbox/abbrvplainnat}
%\bibliography{/home/pvermees/Dropbox/biblio}

\end{document}
