\documentclass{article}
\begin{document}

Thanks for all the background information about \texttt{base64}. Very
interesting! But in the end I have decided to process the raw
\texttt{.RUN} files and bypass \texttt{Pychron} for the time being. As
we discussed in Socorro, I'm not convinced that storing all the raw
data and processing steps on \texttt{GitHub} is a particularly
effective use of a free cloud service.\\

I have now also read Allen's detailed description of your processing
chain, and have a better idea on how I would like to process the
Noblesse data in \texttt{Ar-Ar\_Redux}.  First of all, I really like
your use of a synthetic standard gas mix to calibrate the detectors
and determine the mass fractionation.  It really does kill two birds
with one stone! But I do have one question. You do one peak hop
between:\\

\noindent 101: 40 -- 39 -- 38 -- 37 -- 36, and\\
102: 39 -- 38 -- 37 -- 36 -- 35\\

Why not do the peak hop the other way, i.e.:\\

\noindent 101: 40 -- 39 -- 38 -- 37 -- 36 and\\
102: 41 -- 40 -- 39 -- 38 -- 37\\

This would allow the time zero intercepts to be expressed as logratios
with \textsuperscript{40}Ar as a denominator. The covariance matrix
for the logratio intercepts would then become:

  \[
  \Sigma = \left[
    \begin{array}{cccc}
      \sigma^2[6/0] & cov[6/0,7/0] & 0 & cov[6/0,9/0] \\
      cov[7/0,6/0] & \sigma^2[7/0] & 0 & cov[7/0,9/0] \\
      0 & 0 & \sigma^2[8/0] & 0 \\
      cov[9/0,6/0] & cov[9/0,7/0] & cov[9/0,8/0] & \sigma^2[9/0] \\
    \end{array}
    \right]
  \]

  where `6/0' stands for the
  \textsuperscript{36}Ar/\textsuperscript{40}Ar (log)ratio intercept,
  `7/0' for the \textsuperscript{37}Ar/\textsuperscript{40}Ar
  (log)ratio intercept, etc. From then on data processing could
  proceed exactly as for the Argus-VI.\\

  Using your current routine, I wouldn't be able to use
  \textsuperscript{40}Ar as a normalising isotope, but would need to
  use \textsuperscript{39}Ar instead. That's not an insurmountable
  problem. But it would be more elegant to use a uniform approach for
  all instruments, using the largest signal. This would also result in
  smaller error correlations.

\end{document}
